\section{Introducci�n}
\subsection{Descripci�n del Problema}

\subsection{Funcionalidades a implementar}
Las funcionalidades que se esperan implementadas (SQL/stored
procedures/triggers) son:
\begin{itemize}
 \item Consulta por usuario: obtener, para un usuario espec�fico, informaci�n sobre los art�culos que ha
comprado y vendido, los art�culos que ha visitado con su fecha de visita, los art�culos que tiene
en su lista de favoritos, y las primeras 3 categor�as de art�culos que visit� con mayor frecuencia
en el �ltimo a�o.
 \item Consulta por categor�a de producto: obtener, dada una categor�a de producto (``Computaci�n'',
``Hogar, muebles y jard�n'', etc), un listado de los vendedores que han publicado art�culos de dicha
categor�a y la cantidad de ventas que efectu� cada uno de dichos vendedores.
 \item Funci�n ``Ofertar'': debe permitir al usuario ofertar una suma en una subasta. Dicha suma debe
ser superior en al menos 1 peso, a la oferta actual, e inferior al doble de la oferta actual.
 \item Consulta por usuario y preguntas: obtener para un usuario espec�fico, la lista de preguntas que
ha realizado, con las respectivas respuestas que haya recibido (s�lo la pregunta, si a�n no recibi�
respuesta).
 \item Consulta por keyword: obtener, para un cierto keyword (por ejemplo ``mesa''), la lista de publicaciones
vigentes que tengan en el t�tulo, dicha keyword. El usuario debe poder restringir su
b�squeda s�lo a cierta categor�a de art�culos o servicios.
 \item Consulta por ganador/es anual de viaje a Khan El-Khalili: obtener, para un a�o espec�fico, el ganador/es
\end{itemize}

\section{Modelo de Entidad Relaci�n}
\begin{landscape}
%\includegraphics[scale=.35]{../der.jpg}
\end{landscape}
 
\section{Decisiones tomadas}
\subsection{Decisiones de dise�o}
\begin{description}
 \item[Preguntas a una publicaci�n] Consideramos que una pregunta realizada en una publicaci�n puede tener una �nica respuesta, si el usuario desea replicar la respuesta debe hacerlo mediante una nueva pregunta.
\end{description}

\section{Modelo Relacional}
\begin{description}
 \item[NombreTabla](\underline{pk},\dashuline{fk}, atr1, atr2, ..., atrN)\\
PK=CK=\{(pk)\}\\
FK=\{fk\}\\

\end{description}

\newpage


\section{Restricciones}

La base de datos presenta varias restricciones de lenguaje natural. Las
siguientes restricciones no se encuentran modeladas en la base, y es
responsabilidad de quien ingresa los datos asegurar que se cumplan.

\begin{itemize}
  \item Bla.
  
\end{itemize}

\section{Implementaci�n SQL}
\subsection{Motor elegido}

\subsection{Dise�o de tablas}
\begin{verbatim}
\end{verbatim}


\subsection{Funcionalidades implementadas}
Las funcionalidades implementadas (SQL/stored procedures/triggers) son:
\begin{itemize}
 \item Consulta por usuario: obtener, para un usuario espec�fico, informaci�n sobre los art�culos que ha
comprado y vendido, los art�culos que ha visitado con su fecha de visita, los art�culos que tiene
en su lista de favoritos, y las primeras 3 categor�as de art�culos que visit� con mayor frecuencia
en el �ltimo a�o.
\begin{verbatim}
CREATE bla
\end{verbatim}
  \item Consulta por categor�a de producto: obtener, dada una categor�a de producto (``Computaci�n'',
``Hogar, muebles y jard�n'', etc), un listado de los vendedores que han publicado art�culos de dicha
categor�a y la cantidad de ventas que efectu� cada uno de dichos vendedores.
\begin{verbatim}
CREATE bla
\end{verbatim}
 \item Funci�n ``Ofertar'': debe permitir al usuario ofertar una suma en una subasta. Dicha suma debe
ser superior en al menos 1 peso, a la oferta actual, e inferior al doble de la oferta actual.
\begin{verbatim}
CREATE bla
\end{verbatim}
 \item Consulta por usuario y preguntas: obtener para un usuario espec�fico, la lista de preguntas que
ha realizado, con las respectivas respuestas que haya recibido (s�lo la pregunta, si a�n no recibi�
respuesta).
\begin{verbatim}
CREATE bla
\end{verbatim}
 \item Consulta por keyword: obtener, para un cierto keyword (por ejemplo ``mesa''), la lista de publicaciones
vigentes que tengan en el t�tulo, dicha keyword. El usuario debe poder restringir su
b�squeda s�lo a cierta categor�a de art�culos o servicios.
\begin{verbatim}
CREATE bla
\end{verbatim}
 \item Consulta por ganador/es anual de viaje a Khan El-Khalili: obtener, para un a�o espec�fico, el ganador/es
\end{itemize}

\newpage


\section{Conclusiones}
