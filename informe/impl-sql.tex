\subsection{Motor elegido}

El motor elegido para implementar el sistema de persistencia del mercado virtual fue SQLite\footnote{https://www.sqlite.org/}. Elegimos el mismo dada las facilidades que presenta a la hora del desarrollo de la base de datos, pudiendo almacenar la misma �ntegramente en un archivo standalone que se puede compartir f�cilmente. Adicionalmente, no es necesario configurar un entorno SQL para consultar la base, sino que se puede hacer sencillamente desde un programa de l�nea de comandos de f�cil instalaci�n.

\subsection{Dise�o de tablas}

El dise�o l�gico de las tablas se describe a modo de sentencias SQL de tipo CREATE TABLE, las cuales se encuentran al final de este documento en el ap�ndice \ref{App:AppendixA}.
\subsection{Aclaraciones}

A la hora de implementar la herencia disjunta en las entidades debimos colocar un identificador para saber de que tipo es. \\
Dicho identificador es un INTEGER, a continuaci�n explicamos que implica cada n�mero.\\
\begin{description}

\item[Publicacion] En tipoVigencia, un 0 implica vigente y un 1 finalizada. Para el tipo de publicaci�n, 0 implica onrmal, 1 subasta.

\item[Item] En la entidad Item, 0 implica que es un prooducto y 1 que es un servicio.

\item[Pago] Aqu� un 0 implica pago al contado y 1 pago con tarjeta.

\item[Usuario] Un 0 en esta entidad significa particular y un 1 que es una empresa.

\item[Retiro] En cuanto a Retiros, un 0 implica retiro personal, en cambio, un 1 implica env�o postal.

\end{description}

\subsection{Funcionalidades implementadas}

Las funcionalidades adicionales descriptas en la secci�n \ref{sec:introduccion} se implementaron como consultas en SQL, cuyo c�digo puede encontrarse al final de este documento en el ap�ndice \ref{App:AppendixB}.
