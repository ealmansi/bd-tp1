\subsection{Motor elegido}

\subsection{Dise�o de tablas}
\begin{verbatim}
\end{verbatim}

\subsection{Aclaraciones}

A la hora de implementar la herencia disjunta en las entidades debimos colocar un identificador para saber de que tipo es. \\
Dicho identificador es un INTEGER, a continuaci�n explicamos que implica cada n�mero.\\
\begin{description}

\item[Publicacion] En tipoVigencia, un 0 implica vigente y un 1 finalizada. Para el tipo de publicaci�n, 0 implica onrmal, 1 subasta.

\item[Item] En la entidad Item, 0 implica que es un prooducto y 1 que es un servicio.

\item[Pago] Aqu� un 0 implica pago al contado y 1 pago con tarjeta.

\item[Usuario] Un 0 en esta entidad significa particular y un 1 que es una empresa.

\item[Retiro] En cuanto a Retiros, un 0 implica retiro personal, en cambio, un 1 implica env�o postal.

\end{description}

\subsection{Funcionalidades implementadas}
Las funcionalidades implementadas (SQL/stored procedures/triggers) son:
\begin{itemize}
 \item Consulta por usuario: obtener, para un usuario espec�fico, informaci�n sobre los art�culos que ha
comprado y vendido, los art�culos que ha visitado con su fecha de visita, los art�culos que tiene
en su lista de favoritos, y las primeras 3 categor�as de art�culos que visit� con mayor frecuencia
en el �ltimo a�o.
\begin{verbatim}
CREATE bla
\end{verbatim}
  \item Consulta por categor�a de producto: obtener, dada una categor�a de producto (``Computaci�n'',
``Hogar, muebles y jard�n'', etc), un listado de los vendedores que han publicado art�culos de dicha
categor�a y la cantidad de ventas que efectu� cada uno de dichos vendedores.
\begin{verbatim}
CREATE bla
\end{verbatim}
 \item Funci�n ``Ofertar'': debe permitir al usuario ofertar una suma en una subasta. Dicha suma debe
ser superior en al menos 1 peso, a la oferta actual, e inferior al doble de la oferta actual.
\begin{verbatim}
CREATE bla
\end{verbatim}
 \item Consulta por usuario y preguntas: obtener para un usuario espec�fico, la lista de preguntas que
ha realizado, con las respectivas respuestas que haya recibido (s�lo la pregunta, si a�n no recibi�
respuesta).
\begin{verbatim}
CREATE bla
\end{verbatim}
 \item Consulta por keyword: obtener, para un cierto keyword (por ejemplo ``mesa''), la lista de publicaciones
vigentes que tengan en el t�tulo, dicha keyword. El usuario debe poder restringir su
b�squeda s�lo a cierta categor�a de art�culos o servicios.
\begin{verbatim}
CREATE bla
\end{verbatim}
 \item Consulta por ganador/es anual de viaje a Khan El-Khalili: obtener, para un a�o espec�fico, el ganador/es
\end{itemize}

\newpage
