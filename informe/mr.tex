\begin{description}
% \item[NombreTabla](\underline{pk},\dashuline{fk}, atr1, atr2, ..., atrN)\\
%PK=CK=\{pk\}\\
%FK=\{fk\}\\

 %************* DER USUARIO *************%
 \item[Usuario](\underline{idUsuario}, calle, numero, localidad, telefono, email, tipo)\\
PK=CK=\{idUsuario\}\\
FK=\{\}\\

 \item[Particular](\underline{\dashuline{idUsuario}}, DNI, nombre, apellido)\\
PK=\{idUsuario\}\\
CK=\{idUsuario, DNI\}\\
FK=\{idUsuario\}\\
Particular.idUsuario debe estar en Usuario.idUsuario

 \item[Empresa](\underline{\dashuline{idUsuario}}, CUIT, razonSocial, \dashuline{nombreCategoriaEmp})\\
PK=\{idUsuario\}\\
CK=\{idUsuario, CUIT\}\\
FK=\{idUsuario, nombreCategoriaEmp\}\\
Empresa.idUsuario debe estar en Usuario.idUsuario\\
Empresa.nombreCategoriaEmp debe estar en CategoriaEmpresa.nombre

 \item[CategoriaEmpresa](\underline{nombre}, \dashuline{nombreSubCategoria})\\
PK=CK=\{(nombre)\}\\
FK=\{nombreSubCategoria\}\\
CategoriaEmpresa.nombreSubCategoria puede ser nulo o debe estar en CategoriaEmpresa.nombre\\
CategoriaEmpresa.nombre puede no estar en CategoriaEmpresa.nombreSubCategoria

 \item[SuscripcionRubiOriente](\underline{idSuscripcion}, periodo, \dashuline{idUsuario})\\
PK=CK=\{idSuscripcion\}\\
FK=\{idUsuario\}\\
SuscripcionRubiOriente.idUsuario debe estar en Usuario.idUsuario\\
Usuario.idUsuario puede no estar en SuscripcionRubiOriente.idUsuario

 \item[Factura](\underline{idFactura}, periodo, monto, \dashuline{idUsuario})\\
PK=CK=\{idFactura\}\\
FK=\{idUsuario\}\\
Factura.idUsuario debe estar en Usuario.idUsuario\\
Usuario.idUsuario puede no estar en Factura.idUsuario

 %************* DER PUBLICACION *************%
 \item[Publicacion](\underline{idPublicacion}, título, fecha, precio, \dashuline{tipoPublicacion}, tipoVigencia, tipoVenta, \dashuline{idUsuarioPublicador})\\
PK=CK=\{idPublicacion\}\\
FK=\{tipoPublicacion, idUsuarioPublicador\}\\
Publicacion.tipoPublicacion debe estar en TipoPublicacion.nombre\\
TipoPublicacion.nombre puede no estar en Publicacion.tipoPublicacion

 \item[TipoPublicacion](\underline{idPublicacion}, comision, costo, nivel, caducidad)\\
PK=CK=\{nombre\}\\
FK=\{\}\\

 \item[Item](\underline{idItem}, \dashuline{idPublicacion}, tipo)\\
PK=CK=\{idItem, \dashuline{idPublicacion}\}\\
FK=\{idPublicacion\}\\

 \item[Producto](\underline{\dashuline{idItem}}, \dashuline{nombreCategoriaProd})\\
PK=CK=\{idItem\}\\
FK=\{idItem, nombreCategoriaProd\}\\
Producto.idItem debe estar en Item.idItem\\
Producto.nombreCategoriaProd debe estar en CategoriaProducto.nombre\\
CategoriaProducto.nombre puede no estar en Producto.nombreCategoriaProd

 \item[Servicio](\underline{\dashuline{idItem}}, precioXHora, \dashuline{nombreTipoServicio})\\
PK=CK=\{idItem\}\\
FK=\{idItem, nombreTipoServicio\}\\
Servicio.idItem debe estar en Item.idItem\\
Servicio.nombreTipoServicio debe estar en TipoServicio.nombre\\
TipoServicio.nombre puede no estar en Servicio.nombreTipoServicio

 \item[CategoriaProducto](\underline{nombre}, \dashuline{nombreSuperCategoria})\\
PK=CK=\{(nombre)\}\\
FK=\{nombreSuperCategoria\}\\
CategoriaProducto.nombreSuperCategoria puede ser nulo o debe estar en CategoriaProducto.nombre\\
CategoriaProducto.nombre puede no estar en CategoriaProducto.nombreSuperCategoria

 \item[TipoServicio](\underline{nombre}, \dashuline{nombreSuperTipo})\\
PK=CK=\{(nombre)\}\\
FK=\{nombreSuperTipo\}\\
TipoServicio.nombreSubTipo puede ser nulo o debe estar en TipoServicio.nombre\\
TipoServicio.nombre puede no estar en TipoServicio.nombreSubTipo


 %************* DER COMPRA *************%
 \item[Compra](\underline{idCompra}, \underline{idUsuario}, \underline{idPublicacion})\\
PK=CK=\{idCompra\}\\
FK=\{idUsuario, idPublicacion\}\\

 \item[Pago](\underline{idPago}, monto, \dashuline{idCompra}, tipo)\\
PK=CK=\{idPago\}\\
FK=\{idCompra\}\\

 \item[PagoTarjeta](\underline{\dashuline{idPago}}, númeroTarjeta, cuotas)\\
PK=CK=\{idPago\}\\
FK=\{idPago\}\\

 \item[Calificacion](\underline{idCalificacion}, \dashuline{idCompra}, valoracionComprador, valoracionVendedor, comentarioComprador, comentarioVendedor)\\
PK=CK=\{idCalificacion\}\\
FK=\{idCompra\}\\
Los atributos valoracionComprador, valoracionVendedor, comentarioComprador y comentarioVendedor pueden ser null

 \item[Retiro](\underline{idRetiro}, \dashuline{idCompra}, tipo)\\
PK=CK=\{idRetiro\}\\
FK=\{idCompra\}\\

 \item[EnvioPostal](\underline{\dashuline{idRetiro}}, direccion)\\
PK=CK=\{idRetiro\}\\
FK=\{idRetiro\}\\


 %************* DER CENTRAL *************%
  \item[HizoOferta](\underline{\dashuline{idUsuario}}, \underline{\dashuline{idPublicacion}}, \underline{fecha}, monto)\\
PK=CK=\{(idUsuario, idPublicacion, fecha)\}\\
FK=\{idUsuario, idPublicacion\}\\

% ESTO ES RARO, LA CC ES EL IDPREGUNTA, OSEALAS FOREIGN KEY DEJAN DE IMPORTAR PARA LAS CLAVES CANDIDATAS
% TAL VEZ SEA UNA ENTIDAD, NO UNA RELACION 
  \item[DejoPregunta](\underline{idPregunta}, \dashuline{idUsuario}, \dashuline{idPublicacion}, pregunta, respuesta)\\
PK=CK=\{idPregunta\}\\
FK=\{idUsuario, idPublicacion\}\\

  \item[MarcoFavorita](\underline{\dashuline{idUsuario}}, \underline{\dashuline{idPublicacion}})\\
PK=CK=\{(idUsuario, idPublicacion)\}\\
FK=\{idUsuario, idPublicacion\}\\

  \item[Visito](\underline{\dashuline{idUsuario}}, \underline{\dashuline{idPublicacion}}, \underline{fecha})\\
PK=CK=\{(idUsuario, idPublicacion, fecha)\}\\
FK=\{idUsuario, idPublicacion\}\\

\end{description}

\newpage
