\begin{description}
% \item[NombreTabla](\underline{pk},\dashuline{fk}, atr1, atr2, ..., atrN)\\
%PK=CK=\{pk\}\\
%FK=\{fk\}\\

 %************* DER USUARIO *************%
 \item[Usuario](\underline{idUsuario}, calle, numero, localidad, telefono, email, tipo)\\
PK=CK=\{idUsuario\}\\
FK=\{\}

 \item[Particular](\underline{\dashuline{idUsuario}}, DNI, nombre, apellido)\\
PK=\{idUsuario\}\\
CK=\{idUsuario, DNI\}\\
FK=\{idUsuario\}\\
Particular.idUsuario debe estar en Usuario.idUsuario

 \item[Empresa](\underline{\dashuline{idUsuario}}, CUIT, razonSocial, \dashuline{nombreCategoriaEmp})\\
PK=\{idUsuario\}\\
CK=\{idUsuario, CUIT\}\\
FK=\{idUsuario, nombreCategoriaEmp\}\\
Empresa.idUsuario debe estar en Usuario.idUsuario\\
Empresa.nombreCategoriaEmp debe estar en CategoriaEmpresa.nombre

 \item[CategoriaEmpresa](\underline{nombre}, \dashuline{nombreSubCategoria})\\
PK=CK=\{nombre\}\\
FK=\{nombreSubCategoria\}\\
CategoriaEmpresa.nombreSubCategoria puede ser nulo o debe estar en CategoriaEmpresa.nombre\\
CategoriaEmpresa.nombre puede no estar en CategoriaEmpresa.nombreSubCategoria

 \item[SuscripcionRubiOriente](\underline{idSuscripcion}, periodo, \dashuline{idUsuario})\\
PK=CK=\{idSuscripcion\}\\
FK=\{idUsuario\}\\
SuscripcionRubiOriente.idUsuario debe estar en Usuario.idUsuario\\
Usuario.idUsuario puede no estar en SuscripcionRubiOriente.idUsuario

 \item[Factura](\underline{idFactura}, periodo, monto, \dashuline{idUsuario})\\
PK=\{idFactura\}\\
CK=\{idFactura, (idUsuario, periodo)\}\\
FK=\{idUsuario\}\\
Factura.idUsuario debe estar en Usuario.idUsuario\\
Usuario.idUsuario puede no estar en Factura.idUsuario

 %************* DER PUBLICACION *************%
 \item[Publicacion](\underline{idPublicacion}, titulo, fecha, precio, \dashuline{tipoPublicacion}, tipoVigencia, tipoVenta, \dashuline{idUsuarioPublicador})\\
PK=CK=\{idPublicacion\}\\
FK=\{tipoPublicacion, idUsuarioPublicador\}\\
Publicacion.tipoPublicacion debe estar en TipoPublicacion.nombre\\
TipoPublicacion.nombre puede no estar en Publicacion.tipoPublicacion

 \item[TipoPublicacion](\underline{nombre}, comision, costo, nivel, caducidad)\\
PK=CK=\{nombre\}\\
FK=\{\}

 \item[Item](\underline{idItem}, \dashuline{idPublicacion}, nombre, tipo)\\
PK=CK=\{idItem\}\\
FK=\{idPublicacion\}

 \item[Producto](\underline{\dashuline{idItem}}, \dashuline{nombreCategoriaProd})\\
PK=CK=\{idItem\}\\
FK=\{idItem, nombreCategoriaProd\}\\
Producto.idItem debe estar en Item.idItem\\
Producto.nombreCategoriaProd debe estar en CategoriaProducto.nombre\\
CategoriaProducto.nombre puede no estar en Producto.nombreCategoriaProd

 \item[Servicio](\underline{\dashuline{idItem}}, precioXHora, \dashuline{nombreTipoServicio})\\
PK=CK=\{idItem\}\\
FK=\{idItem, nombreTipoServicio\}\\
Servicio.idItem debe estar en Item.idItem\\
Servicio.nombreTipoServicio debe estar en TipoServicio.nombre\\
TipoServicio.nombre puede no estar en Servicio.nombreTipoServicio

 \item[CategoriaProducto](\underline{nombre}, \dashuline{nombreSuperCategoria})\\
PK=CK=\{nombre\}\\
FK=\{nombreSuperCategoria\}\\
CategoriaProducto.nombreSuperCategoria puede ser nulo o debe estar en CategoriaProducto.nombre\\
CategoriaProducto.nombre puede no estar en CategoriaProducto.nombreSuperCategoria

 \item[TipoServicio](\underline{nombre}, \dashuline{nombreSuperTipo})\\
PK=CK=\{nombre\}\\
FK=\{nombreSuperTipo\}\\
TipoServicio.nombreSuperTipo puede ser nulo o debe estar en TipoServicio.nombre\\
TipoServicio.nombre puede no estar en TipoServicio.nombreSuperTipo


 %************* DER COMPRA *************%
 \item[Compra](\underline{idCompra}, \dashuline{idUsuario}, \dashuline{idPublicacion}, \dashuline{idPago}, \dashuline{idCalificacion}, \dashuline{idRetiro})\\
PK=CK=\{idCompra\}\\
FK=\{idUsuario, idPublicacion, idPago, idCalificacion, idRetiro\}\\
Compra.idUsuario debe estar en Usuario.idUsuario \\
Compra.idPublicacion debe estar en Publicacion.idPublicacion
Compra.idPago debe estar en Pago.idPago \\
Compra.idCalificacion debe estar en Calificacion.idCalificacion \\
%Calificacion.idCalificacion puede no estar en Compra.idCalificacion \\
Compra.idRetiro debe estar en Retiro.idRetiro 

 \item[Pago](\underline{idPago}, monto, tipo)\\
PK=CK=\{idPago\}\\
FK=\{\}

 \item[PagoTarjeta](\underline{\dashuline{idPago}}, numeroTarjeta, cuotas)\\
PK=CK=\{idPago\}\\
FK=\{idPago\}\\
PagoTarjeta.idPago debe estar en Pago.idPago

%  \item[PagoEfectivo](\underline{\dashuline{idPago}})\\
% PK=CK=\{idPago\}\\
% FK=\{idPago\}\\
% PagoEfectivo.idPago debe estar en Pago.idPago

 \item[Calificacion](\underline{idCalificacion}, valoracionComprador, valoracionVendedor, comentarioComprador, comentarioVendedor)\\
PK=CK=\{idCalificacion\}\\
FK=\{\}
%Los atributos valoracionComprador, valoracionVendedor, comentarioComprador y comentarioVendedor pueden ser null\\

 \item[Retiro](\underline{idRetiro}, tipo)\\
PK=CK=\{idRetiro\}\\
FK=\{\}

 \item[EnvioPostal](\underline{\dashuline{idRetiro}}, direccion)\\
PK=CK=\{idRetiro\}\\
FK=\{idRetiro\}\\
EnvioPostal.idRetiro debe estar en Retiro.idRetiro

 %************* DER CENTRAL *************%
  \item[HizoOferta](\underline{\dashuline{idUsuario}}, \underline{\dashuline{idPublicacion}}, \underline{fecha}, monto)\\
PK=CK=\{(idUsuario, idPublicacion, fecha)\}\\
FK=\{idUsuario, idPublicacion\}\\
HizoOferta.idUsuario debe estar en Usuario.idUsuario\\
HizoOferta.idPublicacion debe estar en Publicacion.idPublicacion

  \item[Pregunta](\underline{idPregunta}, \dashuline{idUsuario}, \dashuline{idPublicacion}, pregunta, respuesta)\\
PK=CK=\{idPregunta\}\\
FK=\{idUsuario, idPublicacion\}\\
Pregunta.idUsuario debe estar en Usuario.idUsuario\\
Pregunta.idPublicacion debe estar en Publicacion.idPublicacion\\
Usuario.idUsuario puede no estar en Pregunta.idUsuario\\
Publicacion.idPublicacion puede no estar en Pregunta.idPublicacion

  \item[MarcoFavorita](\underline{\dashuline{idUsuario}}, \underline{\dashuline{idPublicacion}})\\
PK=CK=\{(idUsuario, idPublicacion)\}\\
FK=\{idUsuario, idPublicacion\}\\
MarcoFavorita.idUsuario debe estar en Usuario.idUsuario\\
MarcoFavorita.idPublicacion debe estar en Publicacion.idPublicacion

  \item[Visito](\underline{\dashuline{idUsuario}}, \underline{\dashuline{idPublicacion}}, \underline{fecha})\\
PK=CK=\{(idUsuario, idPublicacion, fecha)\}\\
FK=\{idUsuario, idPublicacion\}\\
Visito.idUsuario debe estar en Usuario.idUsuario\\
Visito.idPublicacion debe estar en Publicacion.idPublicacion

\end{description}

\newpage
